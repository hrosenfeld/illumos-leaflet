\documentclass[11pt,foldmark,notumble]{leaflet}
\renewcommand*\foldmarkrule{.3mm}
\renewcommand*\foldmarklength{5mm}

\usepackage[T1]{fontenc}
\usepackage{german}
\usepackage[utf8]{inputenc}

\usepackage{url}
\usepackage{graphicx}

\title{\bf illumos}

\hyphenation{über-tra-gen OmniOS OmniTI SmartOS Joyent}

%\CutLine*{1}% Dotted line without scissors
%\CutLine*{6}% Dotted line without scissors

\AddToBackground{1}{%  Background of a small page
  \put(30,460){\includegraphics[scale=0.7]{images/LargePhoenixLogotypeRGB.eps}}}


\begin{document}

\vspace*{4cm}
illumos ist ein freies Betriebssystem, welches von OpenSolaris
abgeleitet wurde. Das Projekt wurde im Jahr~2010 gegründet, um
den OpenSolaris-Code unabhängig von Oracle weiterzuentwickeln.

illumos wird von Unternehmen unterstützt, die den illumos-Code als
Grundlage für ihre Produkte verwenden. Das illumos-Projekt dient dabei
der gemeinschaftlichen Entwicklung der Kernbestandteile des
Betriebssystems, welche von den verschiedenen illumos-Distributionen
verwendet werden.

Manche illumos-Distributionen werden von Unternehmen entsprechend
ihren Anforderungen entwickelt und setzen ihre Schwerpunkte z.B. auf
Storage, Virtualisierung oder Cloud Computing. Darüber hinaus gibt
es Community-Distributionen, die für einen allgemeinen Server- und
Desktopeinsatz gedacht sind. Sie alle verwenden den gleichen Kernel,
Bibliotheken, Daemons und Utilities von illumos.


\section{illumos Schlüsseltechnologien}
\subsection{ZFS - Das Zettabyte File System}
ZFS ist ein zukunftssicheres 128-bit Dateisystem, dessen Schwerpunkte
auf garantierter Datenintegrität, Skalierbarkeit, Geschwindigkeit und
einfacher Administration liegen. ZFS macht Volume-Manager oder
RAID-Systeme unnötig.

ZFS verwaltet Festplatten und SSDs in Pools, aus welchen nach Bedarf
Speicher alloziiert wird. Dies vereinfacht die Konfiguration und sorgt
für die Skalierbarkeit, z.B. können Platten einfach hinzugefügt oder
durch Größere ersetzt werden, um die Pool-Kapazität zu steigern.
Redundanz kann in Pools durch Mirroring oder RAIDZ erreicht werden.
Zur deutlichen Steigerung der Geschwindigkeit in Festplatten-Pools
können SSDs als Cache hinzugefügt werden, und natürlich können Pools
komplett aus SSDs aufgebaut werden.

ZFS speichert zu jedem Daten- und Metadaten-Block eine Prüfsumme zur
automatischen Erkennung von Fehlern. Bei vorhandener Redundanz erfolgt
auch eine automatische Korrektur. Dadurch werden Beschädigungen der
Daten durch kurzzeitige Hardwarefehler immer erkannt. Da Blöcke nie
überschrieben werden (Copy-On-Write) und alle Dateisystemoperationen
als Transaktion durchgeführt werden ist die Konsistenz des
Dateisystems und der Daten stets gewährleistet.

Das Erstellen von Dateisystemen, Snapshots und Clones in einem Pool
ist sehr effizient und verbraucht selbst keinen Speicher. Erst wenn
neue Daten geschrieben oder Alte überschrieben werden wird neuer
Speicher vom Pool angefordert.

Der ZFS \emph{ARC}, kurz für ``Adaptive Replacement Cache'', ist ein
moderner RAM-Cache für Daten und Metadaten. Dadurch werden Zugriffe
auf das Dateisystem und Festplatten beschleunigt und die Latenz im
gesamten System reduziert. Der \emph{ARC} kann ausserdem SSDs als
2nd-Level-Cache verwenden um die Cachegröße zu steigern ohne große
Mengen RAM zu verbrauchen.

Durch Verwendung von SSDs als ``ZFS intent log'' (\emph{ZIL}) können
ausserdem synchrone Schreibzugriffe massiv beschleunigt werden.

\subsection{Virtualisierung}
\textbf{Zones} sind eine OS-basierte Virtualisierungstechnik ohne den
Overhead von Para- oder Hardware-Virtualisierung. Die Verwendung von
Systemressourcen wie CPU-Zeit, RAM, Dateisystemen,
Netzwerkschnittstellen und -bandbreite können sehr einfach für jede
Zone begrenzt werden. LX-Branded Zones unter SmartOS erlauben die
native Ausführung unveränderter Linux-Distributionen oder von
Docker-Containern.

Mit \textbf{KVM}, der \emph{kernel virtual machine}, welche durch
Joyent von Linux auf illumos portiert wurde, ist auch die Verwendung
unveränderter Gast-Betriebssysteme möglich. KVM und Zones ergänzen
sich hervorragend: in Joyents SmartOS läuft jede KVM-Instanz in einer
Zone, wodurch die bei Zones übliche einfache Verwaltung und
Ressourcenkontrolle auf KVM anwendbar ist.

Die Verwendung von ZFS-Snapshots und -Clones ermöglicht außerdem die
einfache Erzeugung neuer virtueller Maschinen in Sekundenschnelle.

\subsection{Crossbow -- Netzwerk-Virtualisierung}
Die Netzwerk-Virtualisierungstechnik \emph{Crossbow} ermöglicht es,
auf einem einzigen System virtuelle Netzwerke, Switche and
Netzwerkinterfaces zu erzeugen. Getrennte Netzwerkstacks erlauben
Bandbreitenverwaltung und ermöglichen die Isolation von Protokollen,
Diensten und virtuellen Maschinen. So lassen sich komplizierte
Netzwerktopologien aus Servern, Routern und Firewalls auf einem
einzigen System konsolidieren.

\subsection{Service Management Facility}
Die \emph{Service Management Facility} SMF ersetzt die klassischen
init-Skripte. Sie dient zur einheitlichen und zentralisierten
Konfiguration und Verwaltung von Diensten, und ermöglicht auch die
Überwachung und Steuerung laufender Dienste sowie deren automatischer
Neustart im Fall von Fehlern.

Abhängigkeiten zwischen den Diensten werden von SMF verwaltet,
voneinander unabhängige Dienste werden parallel gestartet.

\subsection{Role-Based Access Control}
\emph{Role-Based Access Control} (RBAC) ist ein Sicherheitsverfahren
zur Zuweisung von eng begrenzte Privilegien und privilegierten Rollen
an Benutzer und Prozesse. Rollen sind eine spezielle Account-Variante,
welche nicht zum Login genutzt werden kann. Stattdessen können Sie von
normalen Benutzern vorrübergehend angenommen werden, wenn die Rolle
ihrem Account zugewiesen wurde. Die einzelnen Privilegien können es
einem Benutzer z.B. erlauben, das System neuzustarten oder einzelne
SMF-Dienste zu verwalten, ohne Befehle mit root-Rechten ausführen zu
müssen.

\subsection{Fault Management Architecture}
Die \emph{Fault Management Architecture} FMA, sammelt, analysiert und
meldet Fehler sowohl von verschiedenen Hardwarekomponenten wie CPUs,
Speicher oder Festplatten, als auch von OS-Subsystemen wie ZFS und
SMF. Die Analyse von Soft-Errors, wie z.B. korrigierbaren ECC-Fehlern
im RAM oder ungewöhnlich langen Zugriffszeiten auf Festplatten,
erlaubt die Vorhersage von Ausfällen und die rechtzeitige
Deaktivierung von Geräten noch bevor ein Ausfall stattfindet.

Fehlerberichte werden dabei in einem konsistenten Format erzeugt und
erlauben dem Systemadministrator, rechtzeitig zu reagieren und
Ausfallzeiten zu minimieren oder sogar zu verhindern.

\subsection{DTrace -- Dynamic Tracing}
DTrace ist ein Instrumentierungs-Werkzeug welches hauptsächlich zur
Performance- und Fehleranalyse eingesetzt wird. Hierzu wird in
laufenden Programmcode dynamisch Instrumentierungscode hineinpatcht.
DTrace kann die gesamte Software auf einem System instrumentieren,
im Userspace genauso wie im Kernel.

DTrace benutzt eine C- und awk-ähnliche Skriptsprache \emph{D},
welche zahlreiche Möglichkeiten zur Filterung und Zusammenfassung
von Daten im Kernel bietet, bevor sie im Userspace weiterverarbeitet
werden. Dies minimiert den Aufwand in der Datensammlung und -analyse
und ermöglicht den sicheren Einsatz von DTrace auf Produktivsystem mit
minimalem Performanceverlust.

Der DTrace-Toolkit ist eine große Sammlung von fertigen
DTrace-Skripten, die fast jeden Bereich der Performance- und
Laufzeit-Analyse abdecken.

Mehr Informationen zu DTrace finden sich auf \textbf{http://dtrace.org}.

\section{Distributionen auf illumos-Basis}
\subsection{OpenIndiana -- http://openindiana.org}
OpenIndiana ist der Nachfolger von Sun's OpenSolaris-Distribution und
eignet sich für den Server- und Desktopeinsatz. Bestehende
OpenSolaris-Installationen können sehr einfach auf OpenIndiana
geupdatet werden. OpenIndiana wird von vielen illumos-Entwicklern als
primäre Entwicklungs- und Testplattform verwendet.

\subsection{NexentaStor -- http://community.nexenta.com}
NexentaStor ist eine NAS/SAN Softwarelösung von Nexenta. Eine
Community-Version ist frei verfügbar, allerdings ohne einige
Management-Features des NexentaStor-Produktes.

\subsection{SmartOS -- http://smartos.org}
SmartOS wird von Joyent speziell für Cloud Computing und
Virtualisierung entwickelt. Das Grundsystem ist auf diesen Zweck
optimiert und kommt ohne Installation oder Paketverwaltung aus. Zur
Bereitstellung zusätzlicher Software in SmartOS-Zones wird der NetBSD
pkgsrc verwendet, an dessen Entwicklung Joyent ebenfalls mitarbeitet.

\subsection{OmniOS -- http://omnios.omniti.com}
OmniOS von OmniTI ist speziell auf den Servereinsatz optimiert. Wie
OpenIndiana verwendet es das Image Packaging System (IPS) von
OpenSolaris, stellt jedoch nur essentielle Pakete bereit. OmniTI
bietet kommerziellen Support für OmniOS an.

\subsection{Weitere illumos-Distributionen}
\begin{itemize}
\item \emph{DilOS} ist eine weitere dpkg/apt-basierte Distribution
\item \emph{Dyson} ist ein Debian-Derivat, welches nur den Kernel, die
  libc und SMF von illumos nutzt
\item \emph{Tribblix} enthält XFCE und E17 als Desktops
\item \emph{XStreamOS} von Sonicle gibt es als Desktop- und Serverversion
\end{itemize}

\subsection{http://illumos.org}
\subsection{discuss@lists.illumos.org}
\subsection{developer@lists.illumos.org}
\subsection{irc.freenode.net: \#illumos}

\end{document}