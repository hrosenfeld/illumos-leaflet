\documentclass[11pt,foldmark,notumble]{leaflet}
\renewcommand*\foldmarkrule{.3mm}
\renewcommand*\foldmarklength{5mm}

\usepackage[T1]{fontenc}
\usepackage[utf8]{inputenc}

\usepackage{url}
\usepackage{graphicx}


\title{\bf illumos}

%\CutLine*{1}% Dotted line without scissors
%\CutLine*{6}% Dotted line without scissors

\AddToBackground{1}{%  Background of a small page
  \put(30,460){\includegraphics[scale=0.7]{images/LargePhoenixLogotypeRGB.eps}}}


\begin{document}
\vspace*{4cm}
illumos is an advanced open source operating system core derived from
OpenSolaris. The project was founded in~2010 to continue the
development of the OpenSolaris code independently of Oracle.

illumos is supported by a few companies who use the illumos code
as a platform for their own products. The illumos project functions as
a hub for collaboration on development of the core operating system
used by the various different illumos distributions.

Some illumos distributions are developed by companies to match their
requirements, and they may focus on storage, virtualization, or cloud
computing, while others are developed by hobbyists and target a more
general server/desktop use case. They all share the common code
consisting of the kernel, libraries, daemons, and utilities.


\section{Core illumos Technologies}
\subsection{ZFS - The Zettabyte File System}
ZFS is a future-proof 128-bit filesystem focusing on guaranteed
end-to-end data integrity, scalability, performance, and ease of use.
ZFS makes volume managers or RAID systems unneccesary.

ZFS manages hard disks and SSDs in storage pools from which storage is
allocated on demand. This simplifies configuration and helps
scalabilty, e.g. disks can simply be added or replaced by larger ones
to increase pool capacity. Redundancy in a pool can be provided by
mirroring or RAIDZ configurations. Using SSDs as cache in hard disk
pools can substantially increase performance, but of course all-SSD
pools are also supported.

ZFS by default stores a checksum for every data and metadata block to
allow automatic detection of errors. Automated error correction is
done when redundancy is available. This eliminates ``silent data
corruption'' caused by transient hardware faults. The consistency of
the filesystem and data is guaranteed since blocks are never
overwritten (copy-on-write) and all file system operations are handled
as transactions.

Creating filesystems, snapshots, and clones in a pool is very
efficient and uses no storage space initially. Storage is only
allocated from the pool when actual new data is written or old data is
changed.

The ZFS \emph{ARC}, short for ``Adaptive Replacement Cache'' is a
modern RAM-backed read cache for data and metadata. This improves file
system and disk performance, driving down overall system latency. It
can also make use of SSDs as 2nd level read cache to increase cache
size without needing massive amounts of RAM.

A separate ``ZFS intent log'' (\emph{ZIL}) stored on fast SSDs can
massively improve the performance of synchronous writes such as done
by NFS servers. Synchronously written data is also temporarily stored
in the ZIL until the relatively slow writes to hard disks are
completed. This allows ZFS to quickly return the synchronous write
calls without compromising reliability.

\subsection{Virtualization}
\textbf{Zones} are an OS-based virtualization technology that does not
have the overhead of paravirtualization or full hardware virtualization.
The use of system resources such as CPU time, memory, storage, network
interfaces, and network bandwidth can be easily limited for each zone.
Special zones called LX-branded zones can run unmodified Linux
distributions directly on an illumos kernel, which is heavily used by
Joyent's SmartOS to run Docker containers natively.

\textbf{KVM}, the \emph{kernel virtual machine}, was ported from Linux
to illumos by Joyent and allows running unmodified guest operating
systems. Both technologies, KVM and zones, can be used together: KVM
instances on SmartOS run in zones, which means that the simple
management and resource control for zones can be applied to KVM.

Both virtualization technologies make great use of ZFS snapshots and
clones, allowing simple and instant creation of new virtual machines.

\subsection{Crossbow -- Network Virtualization}
The network virtualization technology \emph{Crossbow} allows creation
of virtual networks, switches, and network interfaces on a single
system. Separate instances of the network stack allow bandwidth
control and isolation of protocols, services, and virtual machines.
This makes it possible to consolidate complex network topologies
including servers, routers, and firewalls on a single system.

\subsection{Service Management Facility}
The \emph{Service Management Facility} (SMF) replaces the old init
scripts. It allows unified configuration and management of services
in a central place and also provides monitoring and controlling of
running services and automated restarting in the case of errors.

SMF maintains dependencies between services and can start independent
services in parallel.

\subsection{Role-Based Access Control}
\emph{Role-Based Access Control} (RBAC) is a security feature allowing
fine-grained privileges and privileged roles to be assigned to users
and processes. Roles are a special kind of account which cannot be
logged into. Instead they can temporarily be assumed by normal users
that have that role assigned to their account. The individual
privileges may allow a user or role to perform a certain task, like
rebooting the system, or managing a certain SMF service, without the
requirement to allow fully privileged access to certain commands.

\subsection{Fault Management Architecture}
The \emph{Fault Management Architecture} (FMA) collects, analyzes, and
reports errors from different hardware components such as CPUs,
memory, and disks as well as from OS subsystems such as ZFS and SMF.
Analysing soft errors like correctable ECC errors in main memory or
unexpectedly long access times on disks allows prediction of failures
and timely deactivation of devices before catastrophic failure occurs.

Error reports are created in a consistent format and allow the system
administrator to react quickly to minimize or even prevent system
downtime.

\subsection{DTrace -- Dynamic Tracing}
DTrace is an instrumentation tool mostly used for performance analysis
and troubleshooting. To achieve this DTrace dynamically patches live
running instructions with instrumentation code. DTrace can instrument
the whole software stack on a system, programs in user space as well
as the kernel and device drivers.

DTrace uses a scripting language very similar to C and awk, called
\emph{D}, which provides numerous possibilties for filtering and
summarizing data in a safe way in the kernel before handing it off for
more processing in user space. This minimizes the overhead associated
with collecting and presenting data and allows using DTrace
safely and with minimal performance impact on production systems.

The DTrace Toolkit provides a large collection of ready-made scripts
that cover almost all aspects of performance and runtime system
analysis.

More information on DTrace can be found at \textbf{http://dtrace.org}.

\section{Distributions based on illumos}
\subsection{OpenIndiana -- https://www.openindiana.org}
OpenIndiana is the successor to Sun's OpenSolaris distribution and can
be used on both desktop and server systems. It is compatible to
OpenSolaris; existing OpenSolaris installations can be updated to
OpenIndiana very easily. OpenIndiana is used by many illumos
developers as a primary development and test platform.

\subsection{NexentaStor -- https://community.nexenta.com}
NexentaStor is a NAS/SAN software solution developed by Nexenta. A
community edition of NexentaStor is freely available but lacks some
management features of the NexentaStor product.

\subsection{SmartOS -- https://smartos.org}
SmartOS is developed by Joyent specifically for cloud computing and
virtualization purposes. The base system is optimized for this task
and uses no packaging or installation. Additional software for
SmartOS zones is provided through the NetBSD pkgsrc, which is also
worked on by Joyent developers.

\subsection{OmniOS -- https://www.omniosce.org}
OmniOS is opmitized for use on servers. Initially started by OmniTI
it is now maintained by the community. Like OpenIndiana
it uses the Image Packaging System (IPS) from OpenSolaris, but only
essential packages for servers are provided.

\subsection{Other illumos Distributions}
\begin{itemize}
\item \emph{DilOS} is another dpkg/apt based distribution
\item \emph{Dyson} is a Debian derivative that uses only the kernel,
  libc, and SMF from illumos
\item \emph{Tribblix} provides XFCE and E17 desktops
\item \emph{XStreamOS} from Sonicle has variants for desktop and
  server use
\end{itemize}

\subsection{http://illumos.org}
\subsection{discuss@lists.illumos.org}
\subsection{developer@lists.illumos.org}
\subsection{irc.freenode.net: \#illumos}

\end{document}